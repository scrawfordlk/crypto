\documentclass[
    pdftex,
    12pt,
    parskip=half,
    a4paper
]{scrartcl}
\author{Crawford, Sam}
\title{Einführung in Kryptographie und IT-Sicherheit}

\usepackage[utf8]{inputenc}
\usepackage[naustrian]{babel}
\usepackage{graphicx}
\usepackage{listings}
\usepackage{xcolor}

% Define color scheme
\definecolor{codeblue}{rgb}{0.13, 0.13, 0.75}
\definecolor{codegreen}{rgb}{0, 0.5, 0}
\definecolor{codered}{rgb}{0.75, 0.13, 0.13}
\definecolor{codegray}{rgb}{0.5, 0.5, 0.5}
\definecolor{backgray}{rgb}{0.95, 0.95, 0.95}

% Define Python style for listings
\lstdefinestyle{pythonstyle}{
    language=Python,
    basicstyle=\ttfamily\footnotesize,
    keywordstyle=\color{codeblue}\bfseries,
    stringstyle=\color{codered},
    commentstyle=\color{codegreen}\itshape,
    numberstyle=\tiny\color{codegray},
    numbers=left,
    stepnumber=1,
    numbersep=10pt,
    backgroundcolor=\color{backgray},
    showspaces=false,
    showstringspaces=false,
    frame=single,
    tabsize=3,
    breaklines=true,
    breakatwhitespace=true,
}
\lstset{style=pythonstyle}

\usepackage[naustrian]{babel}

\begin{document}
\maketitle

\section{Implementierung der Baker Map}
% \lstinputlisting[style=pythonstyle, caption=This is code]{test.py}

\section{Chaos-basierte Bildverschlüsselung und Entschlüsselung}
\subsection{Motivation}
Für die Verschlüsselung von Bildern wird eine Methode benötigt, die den Inhalt
versteckt. Bilder bestehen aus einer Struktur von Pixeln, d.h. es ist relevant welche Pixel neben welchen
Pixeln liegen. Weiters weisen Bilder eine hohe Redundanz auf. Traditionelle Verschlüsselungsmethoden beachten
diese Aspekte bei Bildern nicht wirklich und somit ist eine andere Verschlüsselungsmethode erforderlich.
Chaos-basierte Verschlüsselung ist deterministisch, hat eine hohe Ergodizidät, d.h. es werden praktisch alle
möglichen Zustände des Systems über einen längeren Zeitraum angenommen, und weist Pseudo-Zufälligkeit auf. Außerdem
sind sie sehr sensibel zu den Ausgangsbedingungen, d.h. kleine Veränderungen der Ausgangsbedingungen führen zu
sehr anderen Ergebnissen. Das bedeutet, dass aufgrund der Komplexität solcher Systeme es schwer ist,
das Verhalten vorherzusehen.
Somit sind chaos-basierte Verschlüsselungen von Vorteil für die Verschlüsselung von Bildern.
\cite{zhang2023}

\subsection{Funktionsweise}
Die chaos-basierte Verschlüsselung nutzt eine chaotische Abbildung wie die Baker Map oder Cat Map. Beide bilden
einen 2-dimensionalen Einheitsquadrat auf sich selber ab. Der Grund warum diese Abbildungen gewählt werden ist, da
sie relativ simpel sind und somit schnell verschlüsselt/enschlüsselt werden kann. 
Eine solche Abbildung wird dann im nächsten Schritt generalisiert, indem Parameter zur Abbildung hinzugefügt werden.
Danach wird sie diskretisiert, denn ein Bild besteht aus diskreten Pixeln. Das bedeutet, dass die Abbildung so
modifiziert wird, sodass sie nicht mehr ein Einheitsquadrat auf sich selbst abbilden, sondern ein 2-dimensionales quadratisches
Bild bestehend aus Pixeln auf sich selber abbildet. So eine Abbildung bestimmt also eine Bijektion zwischen den einzelnen Pixeln
quadratischer Bilder, sodass eine Permutation der Pixel berechnet werden kann. Anschließend wird die Abbildung auf 3 Dimensionen erweitert,
sodass einzelne Pixel Werte modifiziert werden. Schlussendlich wird dann eine Diffusion noch angewendet als Komposition zur bestehenden
Abbildung.
\cite{IEEEMap}

Zur Entschlüsselung wird dann die inverse Baker Map verwendet, sodass man das Ursprungsbild wieder erhaltet nach gleich vielen Iterationen.

\subsubsection{Baker Map}
Die Baker Map ist eine chaotische Abbildung des Einheitsquadrates $I \times I$ auf sich selbst, die formell folgendermaßen beschrieben wird:
$$B(x, y) = (2x, \frac{y}{2}) \text{ if } x < \frac{1}{2}$$
$$B(x, y) = (2x - 1, \frac{y + 1}{2}) \text{ if } \frac{1}{2} \leq x \leq 1$$
Wobei $x$ und $y$ die Koordinaten sind.
In Worten ist die Baker Map eine Abbildung, die einen Einheitsquadrat so auf sich selbst abbildet: Als erstes wird er vertikal in zwei Rechtecken
aufgeteilt. Dann werden diese langgestreckt, sodass die Höhe sich halbiert. Anschließend wird eine der Hälften auf die andere gelegt. Diesen
Prozess kann man mit der Arbeit eines Bäckers beim Kneten von Teig vergleichen, daher der Name.
\cite{IEEEMap}

\subsubsection{Cat Map}
Die standard Cat Map, die ein quadratisches Bild $N \times N$ auf sich selbst abbildet, ist folgendermaßen definiert:
$$
	\begin{bmatrix} x' \\ y' \end{bmatrix} =
	\begin{bmatrix} 1 & 1 \\ 1 & 2 \end{bmatrix}
	\begin{bmatrix} x \\ y \end{bmatrix} (\mod N)
$$
Wobei $x, y \in \{0, 1, \dots , N - 1 \}$, $(x, y)$ ist ein Pixel des Originalbildes und $(x', y')$ ist das abgebildete Pixel.
\cite{catmap}

\section{}
\bibliography{lit}
\bibliographystyle{alpha}
\end{document}
