\documentclass[
    pdftex,
    12pt,
    parskip=half,
    a4paper
]{scrartcl}
\author{Crawford, Sam}
\title{Einführung in Kryptographie und IT-Sicherheit}

\usepackage[utf8]{inputenc}
\usepackage[naustrian]{babel}
\usepackage{graphicx}
\usepackage{listings}
\usepackage{xcolor}

% Define color scheme
\definecolor{codeblue}{rgb}{0.13, 0.13, 0.75}
\definecolor{codegreen}{rgb}{0, 0.5, 0}
\definecolor{codered}{rgb}{0.75, 0.13, 0.13}
\definecolor{codegray}{rgb}{0.5, 0.5, 0.5}
\definecolor{backgray}{rgb}{0.95, 0.95, 0.95}

% Define Python style for listings
\lstdefinestyle{pythonstyle}{
    language=Python,
    basicstyle=\ttfamily\footnotesize,
    keywordstyle=\color{codeblue}\bfseries,
    stringstyle=\color{codered},
    commentstyle=\color{codegreen}\itshape,
    numberstyle=\tiny\color{codegray},
    numbers=left,
    stepnumber=1,
    numbersep=10pt,
    backgroundcolor=\color{backgray},
    showspaces=false,
    showstringspaces=false,
    frame=single,
    tabsize=3,
    breaklines=true,
    breakatwhitespace=true,
}
\lstset{style=pythonstyle}

\usepackage[naustrian]{babel}

\begin{document}
\maketitle

\section{Implementierung der Baker's Map}
% \lstinputlisting[style=pythonstyle, caption=This is code]{test.py}

\section{Chaos-basierte Bildverschlüsselung und Entschlüsselung}
\subsection{Motivation}
Für die Verschlüsselung von Bildern wird eine Methode benötigt, die den Inhalt
versteckt. Bilder bestehen aus einer Struktur von Pixeln, d.h. es ist relevant welche Pixel neben welchen
Pixeln liegen. Weiters weisen Bilder eine hohe Redundanz auf. Traditionelle Verschlüsselungsmethoden beachten
diese Aspekte bei Bildern nicht wirklich und somit ist eine andere Verschlüsselungsmethode erforderlich.
Chaos-basierte Verschlüsselung ist deterministisch, hat eine hohe Ergodizidät, d.h. es werden praktisch alle
möglichen Zustände des Systems über einen längeren Zeitraum angenommen, und weist Pseudo-Zufälligkeit auf. Außerdem
sind sie sehr sensibel zu den Ausgangsbedingungen, d.h. kleine Veränderungen der Ausgangsbedingungen führen zu
sehr anderen Ergebnissen. Das bedeutet, dass aufgrund der Komplexität solcher Systeme es schwer ist,
das Verhalten vorherzusehen.
Somit sind chaos-basierte Verschlüsselungen von Vorteil für die Verschlüsselung von Bildern.
\cite{zhang2023}

\subsection{Funktionsweise}
\cite{IEEEMap}



\subsubsection{Baker Map}

\subsubsection{Cat Map}
\bibliography{lit}
\bibliographystyle{alpha}
\end{document}
